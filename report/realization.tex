\section{Umsetzung}
\subsection{Technologie und Plattform}
In der Problembeschreibung zu dieser Arbeit wurden die Anforderungen an eine SmartHome Lösung diskutiert. In der anschliessenden Marktanalyse wurde openHAB als Grundlage zur Umsetzung unseres Projekts evaluiert. OpenHAB erfüllt die geforderten Kritierien, wie Herstellerunabhängigkeit, Installierbarkeit und Flexibilität. Cloudseitig wird MS Azure Cloud zur Persistierung von Events verwendet.

\subsection{Einführung openHAB}
Das System openHAB wird eingesetzt, um verschiedene Home-Automatisierungssysteme unter einen Hut zu bringen. Um dies zu erreichen müssen Lösungen für die vier Disziplinen Konnektivität, User Interface, Automatisierung und Persistenz gefunden werden. In den nächsten Abschnitten werden diese Disziplinen kurz beschrieben.

\subsubsection{Konnektivität}
Mit Konnektivität ist gemeint, wie die Sensoren/Aktoren integriert werden können. Es braucht ein Konzept um Protokolle miteinander kompatibel zu machen. Nehmen wir als Beispiel den Use Case \emph{L03: Event Kontaktsensor}. An einem Fenster wird ein Kontaktsensor angebracht, der bei jedem Öffnen oder Schliessen den Status bekannt gibt. OpenHAB muss einerseits das verwendete Protokoll verstehen und zudem die Daten in eine interne, abstrakte Form übersetzen, sodass Herstellerspezifische Details vor dem restlichen System verborgen bleiben. Ein weiters Beispiel ist Use Case \emph{L05: Aktor: Lampe}. Hierbei müssen keine Events gelesen, sondern Commands geschickt werden, da es sich bei der Lampe um einen Aktor handelt. Dazu muss openHAB auch den umgekehrten Fall beherrschen, nämlich aus einer internen Repräsentation des Commands in diejenige des Protokolls der Lampe zu übersetzen und letztlich auch die Lampe erreichen können. Durch Konnektivität ist es also möglich, SmartHome Zubehör von verschiedenen Herstellen in openHAB einzubinden. 

\subsubsection{User Interface}
Nehmen wir an, der Fensterkontakt und die Lampe aus dem vorherigen Abschnitt sind von zwei völlig verschiedenen Herstellern. Der Status des Fensterkontakts soll bei der Verwendung ohne openHAB über eine Website im Browser ausgelesen werden können. Das Steuern der Lampe geschehe mittels einer eigens dafür vorgesehenen App. Dank dem Konzept zur Konnektivität können aber beide Geräte auch über openHAB zugegriffen werden. Einen echten Vorteil hat man dadurch aber nur, wenn es auch ein User Interface dazu gibt. Denn dann hat man alle Geräte in einer Smartphone- oder Web App vereint. OpenHAB benötigt demnach eine Möglichkeit um User Interfaces für verschiedene Clients zu gestalten.

\subsubsection{Automatisierung}
Durch die Konnektivität und das User Interface kann also SmartHome Zubehör von verschiedenen Herstellern in einer einzigen Anwendung verwendet werden. Das alleine ist schon ein grosser Mehrwert. Doch es fehlt noch etwas der smarte Teil des SmartHomes. Interessant wird es nämlich dann, wenn die verschiedenen Geräte sich gegenseitig beeinflussen sollen. Nehmen wir den Bewegungsmelder aus Use Case L04 hinzu. Sobald er eine Bewegung registriert soll die Lampe eingeschaltet werden. Es sind aber auch wesentlich komplexere Szenarien denkbar. Damit solche automatisierten Vorgänge stattfinden können benötigt openHAB eine Rule Engine. Die Grundlage dazu bildet die interne Repräsentation der Sensoren und Aktoren, die bereits durch die Konzepte zur Konnektivität geschaffen wurde.

\subsubsection{Persistenz}
Wenn Events gelesen und Commands gesendet werden, dann handelt es sich dabei um Momentaufnahmen. Im User Interface könnte man beobachten, wenn der Status des Fensterkontakts von offen auf zu wechselt. Doch was ist, wenn man wissen möchte, wann das Fenster zuletzt geöffnet wurde? Aus diesem Grund reicht es nicht, Events lediglich zu verarbeiten, sondern sie müssen auch persistiert werden. Zudem können manche Sensoren wie der Fensterkontakt möglicherweise nur immer einen Statuswechsel bekanntgeben, der aktuelle Status kann aber nicht direkt abgefragt werden. Damit trotzdem jederzeit der aktuelle Status bekannt ist, muss openHAB den Status bei jedem Wechsel speichern. 


\subsection{openHAB Architektur}
Im Abschnitt zur Konnektivität haben wir bereits erläutert, welche Anforderungen openHAB erfüllen muss, damit verschiedene Systeme miteinander vernetzt werden können. Die grosse Anzahl an Herstellern und die Vielfalt an Protokollen haben dazu geführt, dass openHAB sehr modular konzipiert wurde. Die Basisinstallation kann zur Laufzeit durch Add-ons erweitert werden. Das hat den Vorteil, dass openHAB selbst recht schlank bleibt und Technologien, die gar nicht eingesetzt werden nicht im Weg sind. Ausserdem ist dadurch das spätere Einbinden weiterer Plattformen sehr einfach machbar. Technisch wurde diese modulare Architektur mit Hilfe der OSGI-Plattform umgesetzt. Die Implementierung von Protokollen geschieht über OSGI Service Bundles, die bei openHAB Bindings genannt werden. Abbildung \ref{fig:ohArch} zeigt einen Überblick der Architektur:

\begin{figure}[H]
	\centering
		\includegraphics[scale=0.45]{report/img/openHAB_architecture}
	\caption{openHAB Architektur}
	\label{fig:ohArch}
\end{figure}

\subsubsection{Bindings}
Bindings sind Verbindungen zwischen openHAB und den externen Systemen und bilden die Grundlage zur Konnektivität. Dadurch muss für jede Technologie ein eigenes Binding geschrieben werden. Für viele Technologien sind Bindings vorhanden, die einzeln heruntergeladen und als «Add-on» installiert werden können. Falls eine Technologie noch nicht unterstützt wird, kann man das Binding dazu selbst programmieren. Die wesentliche Aufgabe besteht darin, sich einerseits mit dem externen Gerät zu verbinden und die ausgetauschten Daten miteinander kompatibel zu machen. \\
Alle momentan verfügbare Bindings sind unter folgendem Link zu finden: \url{https://github.com/openhab/openhab/wiki/Bindings}

\subsubsection{Items}
Wenn so viele unterschiedliche Technologien unterstützt und integriert werden sollen, dann stellt sich die Frage nach dem gemeinsamen Nenner bzw. einer einheitlichen internen Repräsentation. Aus diesem Grund wurden «Items» eingeführt, die zentrale Entität im openHAB Domainmodell. Alle Bindings implementieren ein Mapping zwischen den Daten des Sensors/Aktors und einem zugehörigen Item. Ein Item besteht aus:

\begin{itemize}
	\item Typ
	\item Name
	\item Formatierung
	\item Icon
	\item Gruppe
	\item Bindingparameter
\end{itemize}

Die Eigenschaften Typ und Name sind zwingend, die Anderen sind optional. Der Typ ist auf eine vorgegebene Auswahl beschränkt, der Name dient als Identifier und muss eindeutig sein. 


\subsubsection{Kommunikation}
Der Basisservice von openHAB stellt der Event Bus. Über diesen Bus werden Events zwischen den verschiedenen Bundles gesendet. Die Events sind entweder Commands, welche eine Aktion ausführen, oder Status-Updates, welche Statusänderungen der Devices beinhaltet. \\
Durch den Einsatz dieses EventBus wird die Kopplung reduziert und können somit einfach ausgetauscht werden. \\
Für die Verwaltung der verschiedenen Status ist das Item Repository zuständig, welches permanent den Event Bus auf Status-Updates abhört und die Änderungen ins Repository schreibt. Falls auf einem GUI ein Status eines Devices angezeigt werden soll, kann dazu das Item Repository abgefragt werden.\\
Das Repository persistiert die Status und ist somit auch nach einem Neustart verfügbar.

\begin{figure}[H]
	\centering
		\includegraphics[scale=0.4]{report/img/communicationOH}
	\caption{Kommunikation openHAB}
	\label{fig:ohComm}
\end{figure}

\subsubsection{Bindings}
Ein Binding ist eine Verbindung zwischen dem Event Bus und den externen Systemen. Diese Verbindungen sind aufgrund der verschiedenen Technologien verschieden. Dadurch muss für jede Technologie ein eigenes Binding geschrieben werden. Für einige Systeme sind Bindings vorhanden, die einzeln heruntergeladen und als «Add-on» installiert werden können.
Die Bindings stellen nur sicher, dass Events zwischen Event Bus und den jeweiligen Devices ausgetauscht werden können. Sie müssen sich also nicht um Statusänderungen oder ähnliches kümmern, da dies durch das Item Repository übernommen wird. \\
Alle momentan verfügbare Bindings sind unter folgendem Link zu finden: \url{https://github.com/openhab/openhab/wiki/Bindings}

\pagebreak

\subsection{Installation openHAB}



\subsection{Deploymentübersicht}

\subsubsection{Binding Azure}
\begin{figure}[h!]
	\centering
		\includegraphics[scale=0.5]{report/img/deployment_binding_azure}
	\caption{Binding Azure Cloud}
	\label{fig:deploymentAzure}
\end{figure}
