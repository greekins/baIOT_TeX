\section{Zusammenfassung und Ergebnisse}

Die Aufgabenstellung forderte eine Lösung zur lokalen Vernetzung von Sensoren und Aktoren. Bestehende Systeme auf dem Markt sollten miteinander verglichen und als Lösungsansatz in Betracht gezogen werden. OpenHAB besitzt schon die Fähigkeit zum Vernetzen und intelligenten Steuern von Sensoren und Aktoren. Dadurch konnten wir uns auf den Aufbau eines Demo-Systems mit dem Fokus auf das Szenario Einbruchschutz konzentrieren.

OpenHAB als Basistechnologie einzusetzen erschien uns die geeignetste Wahl, denn es bietet genügend Flexibilät um das Szenario nach unseren Vorstellungen zu realisieren. Dank der Flexibilität konnten wir die Android App und die Cloud-Integration perfekt an das Szenario Einbruchschutz anpassen. Zusammengefasst entstanden während der Arbeit folgende Ergebnisse:
\begin{itemize}
	\item Marktanalyse und Evaluation einer Plattform
	\item Versuchsaufbau mit echter Hardware
	\item OpenHAB Add-On für MQTT in Rules
	\item Cloud-Anwendung zum Sichern von Items
	\item Android App als  openHAB Client
\end{itemize}

Mit openHAB liess sich die Hardware sehr gut in den Versuchsaufbau integrieren. Auch das Zusammenspiel der Hardware funktionierte meist einwandfrei. Man muss kein Software Ingenieur sein, um einen derartigen Versuchsaufbau nachzubilden. Etwas technisches Verständnis und Geduld wird aber vorausgesetzt.

Wir konnten mit dieser Arbeit zeigen, wie openHAB eingesetzt wird und wie ein eigenes Szenario darauf aufgebaut werden kann. Es wurde gezeigt, dass sich openHAB über das MQTT Protokoll mit der Azure Cloud verwenden lässt. 

