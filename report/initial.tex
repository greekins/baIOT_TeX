\section{Ausgangslage}
Diese Bachelorarbeit befasst sich mit einem Teilgebiet des «Internet of Things», nämlich der lokalen Vernetzung von Sensoren und Aktoren.

Gemäss der Aufgabenstellung soll eine SmartHome Beispielapplikation erstellt werden, welche wesentliche Aspekte  einer IoT-Anwendung demostriert. Das beinhaltet das Steuern von Aktoren, Lesen von Sensoren, Event-Verarbeitung, Überwachung und intelligente Abläufe steuern. \\
Das System soll auf einer tragbaren, erweiterbaren Architektur aufgebaut werden und Microsoft Azure als Cloud Plattform benutzen.

Die Heimautomation bzw. ein SmartHome grenzt sich von der professionellen Gebäudeautomation in einigen Aspekten ab. Ein SmartHome, wie wir es umsetzen, umfasst insgesamt deutlich weniger Sensoren und Aktoren, stellt dafür aber höhere Ansprüche an die Installierbarkeit, Bedienbarkeit und niedrige Anschaffungskosten. Unsere Arbeit soll zeigen, welche Überlegungen beim Einstieg in die SmartHome-Welt angestellt werden müssen und auf was für Herausforderungen man dabei stösst.

Die Aufgabenstellung schlägt ein System zum Einbruchschutz als Beispielszenario vor.


\textbf{Hinweis zur Struktur dieses Berichts:}\\
Dieser Bericht basiert gemäss Absprache mit unserem Betreuer auf dem Stukturierungsbeispiel 2. Somit werden die Anforderungen an das System im Kapitel «Problembeschreibung», Architektur- und Design Überlegungen im Kapitel «Lösungskonzept» und Implementierungsdetails, sowie plattformabhängige Entscheidungen im Kapitel «Umsetzung» dokumentiert. Aus diesem Grund sind theoretische Überlegungen zumeist im Lösungskonzept zu finden, das praktische Pendent dazu hingegen im Umsetzungsteil.
