% Der Abstract richtet sich an den Spezialisten auf dem entsprechenden Gebiet
% und beschreibt daher in erster Linie die (neuen, eigenen) Ergebnisse und
% Resultate der Arbeit. Es umfasst nie mehr als eine Seite, typisch sogar nur
% etwa 200 Worte (etwa 20 Zeilen). Es sind keine Bilder zu verwenden.

\chapter*{Abstract}\addcontentsline{toc}{chapter}{Abstract}
Unsere Arbeit befasst sich mit dem Thema «Internet of Things». Das Ziel ist der Aufbau einer Smart-Home Beispielapplikation. Es sollen wesentliche Aspekte einer Internet of Things Anwendung demonstriert werden, die in einem praktischen Szenario denkbar sind. Wir haben uns das Szenario Einbruchschutz erarbeitet und möchten zeigen, wie verschiedene Produkte kombiniert werden können, so wie es in einem Privathaushalt üblich ist. Dazu sollen verschiedene Hardwareteile, Technologien und Frameworks analysiert und in einen Versuchsaufbau integriert werden. Das Szenario umfasst auch eine Cloud-Anwendung, die mit Microsoft Azure umzusetzen ist.
\\ \\
Wir haben entsprechend den Anforderungen das Framework openHAB evaluiert. Aufgrund der modularen Architektur, die auf OSGi basiert, können diverse Protokolle und Features in eine bestehende Installation zur Laufzeit hinzugefügt werden. OpenHAB bezeichnet sich selbst als Platform für das Intranet of Things. Mit unserer MQTT-Verbindung in die Cloud schlagen wir die Brücke zwischen einem lokalen System und den vielseitigen Möglichkeiten der Cloud. Basierend auf der openHAB REST-API wird eine Smartphone App entwickelt, die dem Benutzer den Zugriff auf das System ermöglicht. Der Google GCM-Dienst benachrichtigt den Benutzer mit einer Push- Notification, sobald ein Alarm ausgelöst wurde.
\\ \\
Der Versuchsaufbau besteht aus einem Raspberry Pi als openHAB Server, zwei Philips-Hue Lampen, einem Fensterkontakt, einem Bewegungsmelder, sowie einer Webcam. Während der Entwicklung an der Android App entstand gleichzeitig ein wiederverwendbares SDK für openHAB Apps. Moderne Frameworks, reaktive Programmierung und ein ansprechendes Design runden die User Experience ab. Eine Azure Worker Role fungiert als MQTT-Client und persistiert den Event-Stream von openHAB sofort im Azure Table Storage. Zusammenfassend bietet unser System einen kostengünstigen Einbruchschutz für den Einstieg in die IoT-Smart-Home Welt.