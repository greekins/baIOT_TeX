\chapter{Sitzungsprotokolle}
\label{chap:Sitzungsprotokolle}
\section*{Sitzung 1 - Kick-Off}
	Datum: 17. Februar 2015 \\
	Teilnehmer: Prof. Hansjörg Huser, Dominik Freier, Marco Leutenegger

	\textbf{Projektdefinition} \\
	Aufbauen einer Demoanwendung für «Smart Home». Wie genau die aussehen wird, steht noch nicht fest. Ist Bestandteil der 
	Analyse und Evaluation. Als Resultat der Arbeit soll ein Showcase entstehen 	mit ein paar Anwendungsfällen. \\
	Mögliche Bestandteile:
	\begin{itemize}
		\item Sensoren, Raspberry-Pi
		\item Cloud (simple gehalten, Service Bus)
		\item UI (Mobile/Tablet)
	\end{itemize}

	\textbf{Anstehende Arbeiten}
	\begin{itemize}
		\item Evaluation HW-Platform
		\item Evaluation Framework
		\item erste Version des Projektplans
	\end{itemize}

	\textbf{Organisatorisches}
	\begin{itemize}
		\item Virtueller Server beantragt
		\item Wöchentliche Besprechungen: Mittwoch, 10.10 Uhr
	\end{itemize}

\section*{Sitzung 2}
	Datum: 25. Februar 2015 \\
	Teilnehmer: Prof. Hansjörg Huser, Dominik Freier, Marco Leutenegger

	\textbf{Organisatorisches:}
	\begin{itemize}
		\item System-Architektur von Herrn Huser zur Kenntnis genommen und akzeptiert.
		\item Anwendungsszenarien sollen richtung Einbrecherschutz gehen (Türkontakte, Bewegungssensoren etc.
	\end{itemize}

	\textbf{Anstehende Arbeiten:}
	\begin{itemize}
		\item Bestellliste mit Sensoren und Aktoren erstellen.
		\item Anwendungsszenarien Anpassen.
		\item Erste Version des Projektplans erstellen.
	\end{itemize}

\section*{Sitzung 3}
	Datum: 04. März 2015 \\
	Teilnehmer: Prof. Hansjörg Huser, Dominik Freier, Marco Leutenegger

	\textbf{Organisatorisches:}
	\begin{itemize}
		\item Wunderbar wird vernachlässigt, die Antwort abgewartet. Als Ersatz wird Tinkerforge gewählt.
	\end{itemize}

	\textbf{Anstehende Arbeiten:}
	\begin{itemize}
		\item Bestellliste anpassen.
		\item Einige Änderungen am Projektplan.
		\item Risikoliste anpassen.
		\item Detailplanung erstellen.
	\end{itemize}
	
\section*{Sitzung 4}
	Datum: 11. März 2015 \\
	Teilnehmer: Prof. Hansjörg Huser, Dominik Freier, Marco Leutenegger

	\textbf{Organisatorisches:}
	\begin{itemize}
		\item Iterationsplanung wurde besprochen und gutgeheissen.
		\item Use Cases wurden besprochen und gutgeheissen.
		\item Projektplan wurde als Ganzes angenommen.
		\item Tinkerforge wird nach Absprache nicht weiter verfolgt. Falls genügend Zeit vorhanden ist, werden, aus der bestehenden Hardware, die Use Cases erweitert.
		\item \textbf{Meilenstein 1 erreicht und abgenommen!} - Phase E1 abgeschlossen.
	\end{itemize}

	\textbf{Anstehende Arbeiten:}
	\begin{itemize}
		\item Beginn der Phase E2.
		\item Nach erhalt der Hardware mit Prototyp beginnen.
	\end{itemize}

\section*{Sitzung 5}
	Datum: 17. März 2015 \\
	Teilnehmer: Prof. Hansjörg Huser, Dominik Freier, Marco Leutenegger

	\textbf{Organisatorisches:}

	\textbf{Anstehende Arbeiten:}
	

