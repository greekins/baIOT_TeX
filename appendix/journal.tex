\chapter{Sitzungsprotokolle}
\label{chap:Sitzungsprotokolle}
\section*{Sitzung 1 - Kick-Off}
	Datum: 17. Februar 2015 \\
	Teilnehmer: Prof. Hansjörg Huser, Dominik Freier, Marco Leutenegger

	\textbf{Projektdefinition} \\
	Aufbauen einer Demoanwendung für «Smart Home». Wie genau die aussehen wird, steht noch nicht fest. Ist Bestandteil der 
	Analyse und Evaluation. Als Resultat der Arbeit soll ein Showcase entstehen 	mit ein paar Anwendungsfällen. \\
	Mögliche Bestandteile:
	\begin{itemize}
		\item Sensoren, Raspberry-Pi
		\item Cloud (simple gehalten, Service Bus)
		\item UI (Mobile/Tablet)
	\end{itemize}

	\textbf{Anstehende Arbeiten}
	\begin{itemize}
		\item Evaluation HW-Platform
		\item Evaluation Framework
		\item erste Version des Projektplans
	\end{itemize}

	\textbf{Organisatorisches}
	\begin{itemize}
		\item Virtueller Server beantragt
		\item Wöchentliche Besprechungen: Mittwoch, 10.10 Uhr
	\end{itemize}

\section*{Sitzung 2}
	Datum: 25. Februar 2015 \\
	Teilnehmer: Prof. Hansjörg Huser, Dominik Freier, Marco Leutenegger

	\textbf{Organisatorisches:}
	\begin{itemize}
		\item System-Architektur von Herrn Huser zur Kenntnis genommen und akzeptiert.
		\item Anwendungsszenarien sollen richtung Einbrecherschutz gehen (Türkontakte, Bewegungssensoren etc.
	\end{itemize}

	\textbf{Anstehende Arbeiten:}
	\begin{itemize}
		\item Bestellliste mit Sensoren und Aktoren erstellen.
		\item Anwendungsszenarien Anpassen.
		\item Erste Version des Projektplans erstellen.
	\end{itemize}

\section*{Sitzung 3}
	Datum: 04. März 2015 \\
	Teilnehmer: Prof. Hansjörg Huser, Dominik Freier, Marco Leutenegger

	\textbf{Organisatorisches:}
	\begin{itemize}
		\item Wunderbar wird vernachlässigt, die Antwort abgewartet. Als Ersatz wird Tinkerforge gewählt.
	\end{itemize}

	\textbf{Anstehende Arbeiten:}
	\begin{itemize}
		\item Bestellliste anpassen.
		\item Einige Änderungen am Projektplan.
		\item Risikoliste anpassen.
		\item Detailplanung erstellen.
	\end{itemize}
	
\section*{Sitzung 4}
	Datum: 11. März 2015 \\
	Teilnehmer: Prof. Hansjörg Huser, Dominik Freier, Marco Leutenegger

	\textbf{Organisatorisches:}
	\begin{itemize}
		\item Iterationsplanung wurde besprochen und gutgeheissen.
		\item Use Cases wurden besprochen und gutgeheissen.
		\item Projektplan wurde als Ganzes angenommen.
		\item Tinkerforge wird nach Absprache nicht weiter verfolgt. Falls genügend Zeit vorhanden ist, werden, aus der bestehenden Hardware, die Use Cases erweitert.
		\item \textbf{Meilenstein 1 erreicht und abgenommen!} - Phase E1 abgeschlossen.
	\end{itemize}

	\textbf{Anstehende Arbeiten:}
	\begin{itemize}
		\item Beginn der Phase E2.
		\item Nach erhalt der Hardware mit Prototyp beginnen.
	\end{itemize}

\section*{Sitzung 5}
	Datum: 17. März 2015 \\
	Teilnehmer: Prof. Hansjörg Huser, Dominik Freier, Marco Leutenegger

	\textbf{Organisatorisches:}
	\begin{itemize}
		\item \textbf{Beginn der Phase E2}
		\item Fassung für Glühbirne kaufen die Studenten in einem Elektro-Fachgeschäft. Die Kosten werden vom Institut übernommen.
		\item Router wird von Herrn Huser organisiert.
	\end{itemize}

	\textbf{Anstehende Arbeiten:}
	\begin{itemize}
		\item Installation openHAB auf Raspberry Pi.
		\item Einrichten des Netzwerkes (Router).
		\item Use Cases umsetzen mit HomeMatic, Philips Hue und Webcam.
		\item Erste Version der Architekturdokumentation erstellen.
	\end{itemize}

\section*{Sitzung 6}
	Datum: 25. März 2015 \\
	Teilnehmer: Prof. Hansjörg Huser, Dominik Freier, Marco Leutenegger

	\textbf{Organisatorisches:}
	\begin{itemize}
		\item Prototyp wurde vorgeführt und für gut befunden.
		\item Man befindet sich im Zeitplan und beginnt mit den nächsten Arbeitspaketen.
	\end{itemize}

	\textbf{Anstehende Arbeiten:}
	\begin{itemize}
		\item Termin für Zwischenpräsentation mit Prof. Dr. Rinkel in zwei Wochen (8. April 2015)
		\item Integration Webcam abschliessen.
		\item Mit Aufsetzen von Azure Cloud und Binding beginnen.
	\end{itemize}
	
\section*{Sitzung 7}
	Datum: 01. April 2015 \\
	Teilnehmer: Prof. Hansjörg Huser, Dominik Freier, Marco Leutenegger

	\textbf{Organisatorisches:}
	\begin{itemize}
		\item Cloud: Anstatt ein eigenes Binding zu schreiben, wird das MQTT-Binding für die Kommunikation mit der Cloud eingesetzt.
		\item In der Dokumentation soll mehr auf Übersicht geachtet werden (Big Picture). Am Anfang abstrakt beginnen und immer detailierter werden.
	\end{itemize}

	\textbf{Anstehende Arbeiten:}
	\begin{itemize}
		\item Zwischenpräsentation vorbereiten.
		\item Cloud Aufsetzen und anpassen.
		\item Dokumentation nachführen.
		\item MQTT-Broker evaluieren.
	\end{itemize}
	
\section*{Sitzung 8 (Zwischenpräsentation)}
	Datum: 08. April 2015 \\
	Teilnehmer: Prof. H. Huser, Prof. Dr. A. Rinkel, Dominik Freier, Marco Leutenegger 
	
	Diese Sitzung wurde durch die Zwischenpräsentation für Herr Rinkel ersetzt. Gezeigt wurden alle Aspekte der Bachelorarbeit und der bestehende Prototyp.

	\textbf{Anforderungen betreffend Dokumentation:}
	\begin{itemize}
		\item Event Driven Design: Differenzierung/Abgrenzung.
		\item Möglichkeiten/Grenzen von openHAB aufzeigen.
		\item Einsatz von Cloud begründen, Mehrwert aufzeigen.
		\item Ersetzen von Mobile App begründen, Mehrwert aufzeigen.
	\end{itemize}

	\textbf{Anstehende Arbeiten:}
	\begin{itemize}
		\item Cloud aufsetzen und anpassen.
		\item MQTT-Broker evaluieren.
		\item Integration openHAB in Azure Cloud.
		\item Dokumentation nachführen.
	\end{itemize}

\section*{Sitzung 9}
	Datum: 15. April 2015 \\
	\\Diese Sitzung wurde aufgrund der HSR Stellenbörse und der fehlenden Notwendigkeit einer Standortbestimmung abgesagt.
	
\section*{Sitzung 10} (Ausgefallen)
	Datum: 24. April 2015 \\
	\\Diese Sitzung wurde aufgrund von Terminen von Herrn Huser abgesagt. Anstelle des Meetings wurde der aktuelle Stand per E-Mail festgehalten:
	\begin{quote}
	Folgende Arbeitsschritte wurden gemäss unserem Projektplan verrichtet:
	\begin{itemize}
		\item Einarbeitung MQTT
		\item Evaluation MQTT Broker \& .NET Library
		\item Aufsetzen des MQTT Brokers auf einer VM in der Azure-Cloud (inkl. End-to-end Verschlüsselung)
		\item Erstellen eines Cloudservices mit einer Worker Role, die sich auf Topics subscribed und die empfangenen Messages im Table- bzw. Blob-Storage persistiert.
		\item Erstellen einer Action in openHAB, um die Bilder der Webcam per MQTT an den Broker zu senden.
	\end{itemize}
	Momentan sind wir im Plan eine Woche im Voraus und können ab nächster Woche bereits mit der Mobile-App beginnen.	
	\end{quote}
	
\section*{Sitzung 11}
	Datum: 29. April 2015 \\
	Teilnehmer: Prof. Hansjörg Huser, Dominik Freier, Marco Leutenegger

	\textbf{Organisatorisches:}

	\textbf{Anstehende Arbeiten:}

\section*{Sitzung 12}
	Datum: 24. April 2015 \\
	Teilnehmer: Prof. Hansjörg Huser, Dominik Freier, Marco Leutenegger

	\textbf{Organisatorisches:}

	\textbf{Anstehende Arbeiten:}
	
\section*{Sitzung 12}
	Datum: 06. Mai 2015 \\
	Teilnehmer: Prof. Hansjörg Huser, Dominik Freier, Marco Leutenegger

	\textbf{Organisatorisches:}

	\textbf{Anstehende Arbeiten:}