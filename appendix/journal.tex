\chapter{Sitzungsprotokolle}
\label{chap:Sitzungsprotokolle}
\section{Sitzung 1 - Kick-Off}
Datum: 17. Februar 2015 \\
Teilnehmer: Prof. Hansjörg Huser, Dominik Freier, Marco Leutenegger

\textbf{Projektdefinition} \\
Aufbauen einer Demoanwendung für «Smart Home». Wie genau die aussehen wird, steht noch nicht fest. Ist Bestandteil der Analyse und Evaluation. Als Resultat der Arbeit soll ein Showcase entstehen mit ein paar Anwendungsfällen. \\
Mögliche Bestandteile:
\begin{itemize}
	\item Sensoren, Raspberry-Pi
	\item Cloud (simple gehalten, Service Bus)
	\item UI (Mobile/Tablet)
\end{itemize}

\textbf{Anstehende Arbeiten}
\begin{itemize}
	\item Evaluation HW-Platform
	\item Evaluation Framework
	\item erste Version des Projektplans
\end{itemize}

\textbf{Organisatorisches}

\begin{itemize}
	\item Virtueller Server beantragt
	\item Wöchentliche Besprechungen: Mittwoch, 10.10 Uhr
\end{itemize}

\section{Sitzung 2}
Datum: 25. Februar 2015 \\
Teilnehmer: Prof. Hansjörg Huser, Dominik Freier, Marco Leutenegger

\textbf{Organisatorisches:}
\begin{itemize}
	\item System-Architektur von Herrn Huser zur Kenntnis genommen und akzeptiert.
	\item Anwendungsszenarien sollen richtung Einbrecherschutz gehen (Türkontakte, Bewegungssensoren etc.
\end{itemize}


\textbf{Anstehende Arbeiten:}
\begin{itemize}
	\item Bestellliste mit Sensoren und Aktoren erstellen.
	\item Anwendungsszenarien Anpassen
\end{itemize}
