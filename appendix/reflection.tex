\pagebreak
\section{Persönliche Reflektion}

\subsection*{Marco Leutenegger}
Nach einer erfolgreichen Semesterarbeit durften wir die Bachelorarbeit wieder bei Herrn Prof. Hansjörg Huser machen, was mich persönlich sehr freute. Die Arbeit mit Herrn Huser war sehr interessant und die Zusammenarbeit stets unkompliziert, was ich sehr schätzte.

In der Semesterarbeit stand die Cloud bzw. unser Cloudservice im Fokus. Der Teil des Smart-Homes wurde dabei simuliert. In der Bachelorarbeit konnten wir uns nun diesem Bereich widmen. Dies stellte zu Beginn der Arbeit einige Herausforderungen dar. Einerseits, weil ich mich bis jetzt noch nie mit der Entwicklung zusammen mit Hardware auseinandergesetzt habe. Andererseits war die Aufgabenstellung bezüglich des Szenarios sehr offen. Das heisst, wir mussten uns ein realistisches Konzept erarbeiten, wie Sensoren und Aktoren in einem Haus eingesetzt werden können. Des Weiteren mussten wir evaluieren was für Technologien und Frameworks zur Heimautomation verwendet werden sollen. Dabei mussten wir auch beachten, wie kompatibel die Hardware mit den Frameworks ist. \\
Wir konnten uns dann für das Szenario «Einbruchschutz» festlegen. Das entsprach auch meiner Vorstellung eines sinnvollen Einsatzes von Things, im Rahmen des Internets der Dinge. Wir durften uns darauf hin Hardware im Wert von ca. 500 Franken beschaffen, auf Kosten des INS. Das ist nicht selbstverständlich und spricht für das Vertrauen, das uns entgegengebracht wurde.

Das Highlight für mich war der Zeitpunkt, als das System aufgebaut war und die Item-Updates über MQTT im Table Storage der Azure Cloud erfolgreich persistiert wurden (inklusive des Bildes der Überwachungskamera). Das spezielle an dieser Lösung war die Kombination der Cloud mit der openHAB Plattform.

Nach diesem Teil der Arbeit stand uns eine letzte grosse Hürde bevor, nämlich das Schreiben einer eigenen Android App. Da ich noch nie mit Mobileapplikationen arbeitete, war das ganz neu für mich. Mit der Unterstützung von Dominik Freier, der bereits Erfahrungen in diesem Bereich hatte, fiel die Einarbeit in diese Thematik um einiges leichter und die Freude am fertiggestellten App war gross.

Ich kann nun auf eine sehr gelungene und interessante Bachelorarbeit zurückblicken, die mir viel Spass bereitet hat.

\pagebreak

\subsection*{Dominik Freier}
Zu Beginn der Bachelorarbeit stand uns ein sehr breites Themengebiet offen. Wir hatten das Privileg, selbst mitbestimmen zu dürfen, wohin sich die Arbeit entwickeln soll und wo wir die Schwerpunkte setzen möchten. In der ersten Zeit entwickelten wir zusammen mit unserem Betreuer Herrn Prof. Hansjörg Huser das Smart-Home Szenario Einbruchschutz und definierten die genauen Anforderungen. 

Nach einer Analyse des Marktes erkannten wir, dass es keinen Sinn machen würde selbst eine Lösung zum Integrieren und Kombinieren von Hardware zu entwickeln. OpenHAB passte am besten zu unseren Anforderungen, denn es hielt für alle wichtigen Kriterien einen Lösungsweg bereit. Andere Smart-Home Lösungen auf dem Markt sind zwar ausgereifter, sind aber zu geschlossen und geben das Szenario in den meisten Fällen vor. Bei der Auswahl von Hardware war mir wichtig, dass wir funktionierende Sensoren und Aktoren verwenden, die schon erprobt sind und in Privathaushalten eingesetzt werden. In unserer Arbeit konnten wir demonstrieren, wie sich Herstellergrenzen überwinden lassen. Ich bin der Meinung, dass unser Arbeitsergebnis eine gute Balance zwischen vorgegebenem Szenario und Flexibilität getroffen hat. Einen Mehrwert gegenüber einer reinen openHAB Installation haben wir durch die Cloud-Anbindung und die Android App geschaffen. Obwohl ich mit unserer Arbeit sehr zufrieden bin, möchte ich unser Demo-System nicht als Ersatz für eine professionelle Alarmanlage bezeichnen. Für den produktiven und verlässlichen Einsatz müssten weitere Randbedingungen und Situationen getestet und implementiert werden.

Für den Erfolg dieser Arbeit war die gute Zusammenarbeit zwischen Marco Leutenegger und mir ein entscheidender Faktor. Wir hatten von Anfang an dieselbe Vision und konnten unsere persönlichen und technischen Fähigkeiten ideal einsetzen. Durch paralleles Arbeiten und gute Planung hatten wir keine Stillstände. Trotzdem synchronisierten wir uns täglich über Fortschritte und überprüften die geleisteten Arbeiten gegenseitig. 

Ich fand es etwas schade, dass wir unsere Lösung aufgrund der HSR Richtlinien nicht zu den gleichen Bedingungen wie in einem Privathaushalt aufbauen konnten. Wir konnten nicht ausprobieren, wie man per Remote auf den Versuchsaufbau zugreifen kann, obwohl dies für ein realistisches Szenario wichtig gewesen wäre.

Die Bachelorarbeit war eine sehr lehrreiche und spannende Erfahrung mit einem Themengebiet, das ich mir jederzeit erneut aussuchen würde.

