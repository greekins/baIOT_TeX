\pagebreak
\section{Persönliche Reflektion}

\subsection*{Marco Leutenegger}
Nach einer erfolgreichen Semesterarbeit durften wir die Bachelorarbeit wieder bei Herrn Prof. Hansjörg Huser machen, was mich persönlich sehr freute. Die Arbeit mit Herrn Huser war sehr interessant und die Zusammenarbeit unkompliziert, was ich sehr schätzte.

In der Semesterarbeit stand die Cloud bzw. unser Cloudservice im Fokus. Der Teil des Smart-Homes wurde dabei simuliert. In der Bachelorarbeit konnten wir nun den Smart-Home-Bereich realisieren. Dies stellte zu Beginn der Arbeit einige Herausforderungen dar. Einerseits, weil Dominik Freier und ich bis jetzt noch nie Hardwarenahe entwickelt haben. Andererseits war die Aufgabenstellung bezüglich des Szenarios sehr offen. Das heisst, wir mussten uns ein realistisches Konzept erarbeiten, wie Sensoren und Aktoren eingesetzt werden können. Des weiteren mussten wir evaluieren was für Technologien und Frameworks zur Heimautomation eingesetzt werden sollen. Dabei musste auch beachtet werden, welche Produkte diese Frameworks unterstützen. \\
Wir konnten uns dann für das Szenario «Einbruchschutz» festlegen. Das entsprach auch meiner Vorstellung eines sinnvollen Einsatzes von Things, im Rahmen des Internets der Dinge. Wir durften uns darauf hin Hardware im Wert von ca. 500 Franken beschaffen, auf Kosten des INS. Das ist nicht selbstverständlich und spricht für das Vertrauen, das uns entgegengebracht wurde.

Das Highlight für mich war der Zeitpunkt, als das System aufgebaut war und die Item-Updates über MQTT im Table Storage der Azure Cloud erfolgreich persistiert wurden (inklusive des Bildes der Überwachungskamera). Zu diesem Zeitpunkt entstand ein Prototyp, der in dem Stadium bereits einsatzfähig war.

Nach diesem Teil der Arbeit stand uns eine letzte grosse Hürde bevor, nämlich das Schreiben einer eigenen Android App. Da ich noch nie mit Mobileapplikationen arbeiete, war das ganz neu für mich. Mit der Unterstützung von Dominik Freier, der bereits Erfahrungen in diesem Bereich hatte, fiel die Einarbeit in diese Thematik um einiges leichter und die Freude am fertiggestellten App war riesig.

Ich kann auf eine sehr gelungene Bachelorarbeit zurückblicken, die mir viel Spass bereitete. Der enge Bezug zur Wirklichkeit hat viel dazu beigetragen.

\pagebreak

\subsection*{Dominik Freier}
\tbd
Lorem ipsum dolor sit amet, consetetur sadipscing elitr, sed diam nonumy eirmod tempor invidunt ut labore et dolore magna aliquyam erat, sed diam voluptua. At vero eos et accusam et justo duo dolores et ea rebum. Stet clita kasd gubergren, no sea takimata sanctus est Lorem ipsum dolor sit amet. Lorem ipsum dolor sit amet, consetetur sadipscing elitr, sed diam nonumy eirmod tempor invidunt ut labore et dolore magna aliquyam erat, sed diam voluptua. At vero eos et accusam et justo duo dolores et ea rebum. Stet clita kasd gubergren, no sea takimata sanctus est Lorem ipsum dolor sit amet.