\chapter{Projektplan}

\begin{center}
	\includegraphics[scale=0.75]{appendix/img/openhabLogo}
\end{center}
\vfill
\begin{description}[style=multiline,leftmargin=3cm]
\item[Thema] Aufbau einer Smart-Home Beispielapplikation
\item[Studenten] Dominik Freier, Marco Leutenegger
\item[Betreuer] Prof. Hansjörg Huser
\end{description}
\pagebreak

\section*{Änderungsgeschichte}
	\begin{tabularx}{\textwidth}{lllX}
		\textbf{Datum} & \textbf{Version} & \textbf{Änderung}	 & \textbf{Autor} 
		\\ \hline
			25.02.2015 &
			0.0.1 &
			Dokument erstellen &
			M. Leutenegger
		\\ \hline
			27.02.2015 &
			0.0.2 &
			Meilensteine erfasst	&
			D. Freier, M. Leutenegger
		\\ \hline
			04.03.2015 &
			0.1.0 &
			Risikomanagement angepasst &
			D. Freier, M.Leutenegger
		\\ \hline
	\end{tabularx}
\pagebreak

\section*{Einführung}
	\subsection*{Zweck}
		Dieses Dokument dient als Projektplan für die Bachelorarbeit von Dominik Freier und Marco Leutenegger und definiert alle organisatorischen Rahmenbedingungen.

	\subsection*{Gültigkeitsbereich}
		Die Gültigkeit des Projektplans beschränkt sich auf die Bachelorarbeit von Dominik Freier und Marco Leutenegger im Frühjahrssemester 2015.

	\subsection*{Referenzen}
		\begin{tabularx}{\textwidth}{lX}
			\textbf{Bezeichnung}	& \textbf{Referenz}
			\\ \hline
				Risikomanagement	 &
				Siehe separates Dokument
			\\ \hline
				Security Infos &
				\url{https://github.com/openhab/openhab/wiki/Security}
			\\ \hline
		\end{tabularx}
\pagebreak

\section*{Projekt und Übersicht}
	\subsection*{Zweck und Ziel}
		Diese Bachelorarbeit hat as Ziel, eine Smart-Home Beispielapplikation aufzubauen, 
		welche wesentliche Aspekte einer Internet-of-Things-Anwendung demonstriert, wie 
		Steuern von Devices, Lesen von Sensoren, Event-Verarbeitung, Überwachung und 
		intelligente Abläufe steuern, Streaming von Sensordaten und Online-Analyse der 
		Daten usw. \\ Das System soll auf einer tragfähigen und erweiterbaren Architektur 
		aufgebaut werden und Microsoft Azure als Cloud Plattform benutzen.

	\subsection*{Lieferumfang}
		Die abzuliefernden Dokumente und Software-Artefakte des Projekts richten sich im Wesentlichen 
		nach den Vorgaben aus den Dokumentationsanleitungen der HSR. Eine davon abweichender Lieferumfang 
		wurde mit dem Betreuer besprochen und genehmigt.

	\subsection*{Referenzen}
		\begin{tabularx}{\textwidth}{llXll}
			\textbf{Nr.}	& \textbf{Art} & \textbf{Bezeichnung} & \textbf{Form} & \textbf{Empfänger}
			\\ \hline
				1 &
				Publikation &
				Poster &
				PDF &
				H.Huser
			\\ \hline
				2 &
				Publikation &
				Kurzfassung &
				PDF &
				H.Huser
			\\ \hline
				3 &
				Dokument	 &
				Bericht &
				PDF/Ausdruck	 &
				H.Huser
			\\ \hline
				4 &
				Dokument	 &
				Projektplan &
				PDF/Ausdruck &
				H.Huser
			\\	\hline
				5 &
				Dokument	 &
				Sitzungsprotokolle &
				PDF/Ausdruck &
				H.Huser
			\\ \hline
				6 &
				Dokument &
				Eigenständigkeitserklärung &
				PDF/Ausdruck	 &
				H.Huser
			\\ \hline
				7 &
				Dokument &
				Erfahrungsbericht D.Freier &
				PDF/Ausdruck &
				H.Huser
			\\ \hline
				8 &
				Dokument &
				Erfahrungsbericht M.Leutenegger &
				PDF/Ausdruck &
				H.Huser
			\\ \hline
				9 &
				Source &
				Code-Abgabe &
				ZIP &
				H.Huser
			\\ \hline
				10 &
				Archiv &
				2x Deliverables 1-9 &
				DVD &
				H.Huser
			\\ \hline
		\end{tabularx}
\pagebreak


\section*{Projektorganisation}
	Die Dokumentation des Projekts gliedert sich in diesen Projektplan und einen Bericht. Im 
	Projektplan werden alle organisatorischen Aspekte festgehalten, wie etwa die Planung der 
	Meilensteine, Aufgaben der Teammitglieder oder Abmachungen zum Dokumentemanagement. Im 
	Bericht werden technische Beschreibungen der Ausgangslage, Diskussionen für Lösungsansätze, 
	Requirements und Details zur Umsetzung dokumentiert.
	\\ \\
	Damit die Teammitglieder möglichst parallel und effizient arbeiten können, werden alle Dokumente 
	mit LaTeX geschrieben und auf einem Git-Repository verwaltet. Daruch wird das Risiko von Versionskonflikten 
	reduziert und der Zugriff insbesondere für den Betreuer vereinfacht.
	\\ \\
	Die Verwaltung der Aufgaben und agilen Vorgänge erfolgt durch Jira. Wir erhielten zu diesem Zweck eine 
	Classroom Lizenz vom Hersteller Atlassian. Jira wurde auf einem virtuellen Server der HSR installiert.

	\subsection*{Organisationsstruktur}
		\begin{tabularx}{\textwidth}{Xl}
			\textbf{Verantwortung}	& \textbf{Teammitglied} \\
			\hline
			Verwaltung und Bereinigung de Dokumente		& D. Freier, M. Leutenegger \\ \hline
			Pflege virtueller Server, Jira inkl. Backups	& D. Freier, M. Leutenegger \\ \hline
			Sitzungsprotokolle verfassen					& D. Freier, M. Leutenegger \\ \hline
			Iterationsplanung							& D. Freier, M. Leutenegger \\ \hline
			Risikomanagement								& D. Freier, M. Leutenegger \\ \hline
			Architekturdesign							& D. Freier, M. Leutenegger \\ \hline
		\end{tabularx}

	\subsection*{Externe Schnittstellen}
		Betreuer der Bachelorarbeit ist Prof. Hansjörg Huser. Experte ist Herr Stefan Zettel. Gegenleser ist \tbd .
\pagebreak

\section*{Management Abläufe}
	\subsection*{Zeitliche Planung}
		Das Projekt wird während des Frühjahrssemester 2015 durchgeführt. Der Start der Arbeit war am Montag, 
		den 16. Februar 2015. Die Abgabe der Vollständigen Dokumentation an den Betreuer erfolgt am Freitag, 
		den 12. Juni 2015. Als Zeitbudget sollen in den 17 Wochen insgesamt 720 Stunden, bzw. rund 21 Stunden 
		pro Woche und Student eingeplant werden.

	\subsection*{Vorgehensmodell}
		Als Vorgehensmodell wurde der Rational Unified Process ausgewählt, da das Projektteam mit diesem Modell 
		aus früheren Arbeiten (inkl. Semesterarbeit) vertraut ist und damit gute Erfahrungen gemacht hat. Die 
		Phasen wurden nach dem Schema «eins, drei, drei, eins» in insgesamt acht Iterationen à zwei Wochen aufgeteilt.

	\subsection*{Meilensteine}
		\begin{tabularx}{\textwidth}{lp{1cm}Xl}
			\textbf{MS}	& \textbf{Iter.} & \textbf{Beschreibung} & \textbf{Datum}
			\\ \hline
				MS1 &
				I1 &
				Der Projektauftrag wurde zusammen mit dem Betreuer besprochen und ist akzeptiert. 
				Den Teammitgliedern ist klar, welches die Ziele des Projekts sind und haben eine gemeinsame Vision. 
				Die organisatorischen Aspekte wurden so weit wie möglich abgeklärt und die benötigte Infrastruktur 
				steht allen Beteiligten zur Verfügung. &
				04.03.2015
			\\ \hline		
				MS2 &
				E1 &
				Die Analyse der funktionalen und nicht-funktionalen Anforderungen ist abgeschlossen und die Use Cases definiert. 
				Die technische Umsetzung der Use Cases wurde analysiert und mit Umsetzung kann begonnen werden. Die Hardware wurde 
				bestellt und für das Mobile-App wurden erste Mockups gezeichnet. &
				18.03.2015
			\\ \hline
				MS3 &
				E2 &
				Ein Architekturprototyp (Installation und Konfiguration openHAB) existiert. Ein Prototyp für die Use Cases mit 
				existierenden Bindings wurde entwickelt.	& 01.04.2015
			\\ \hline
				MS4 &
				E3 &
				Prototyp mit eigenen Bindings wurde entwickelt, parallel dazu wird die Cloud mit den benötigten Komponenten aufgesetzt. &
				15.04.2015
			\\ \hline
 				MS5 &
 				C1 &
 				Die Use Cases mit den eigenen Bindings sind fertig implementiert. &
 				29.04.2015
 			\\ \hline
				MS6	&
				C2 &
				Das Android-App ist gemäss den, in der Analyse (E1) gezeichneten, Mockups entwicklt und die geplanten Funktionen sind implementiert. &
				13.05.2015
			\\ \hline
 				MS7 &
 				C3 &
 				Der geschriebene Code wurde überarbeitet und optimiert. Die nötigen Komponenten sind gemäss FR und NFR getestet. &
 				27.05.2015
 			\\ \hline
				MS8 &
				T1 &
				Die Dokumentation wurde nachgeführt, und finalisiert. Die Deliverables werden am darauf folgenden Freitag den entsprechenden 
				Personen übergeben. Dieser Meilenstein definiert den Abschluss des Projektes. &
				10.06.20115
			\\ \hline
		\end{tabularx}

\subsection*{Iterationsplanung}
	\begin{tabularx}{\textwidth}{lXXl}
		\textbf{It.}	&	\textbf{Arbeitspakete}	&	\textbf{Ziele}	&	\textbf{SW}
		\\ \hline
			I1	&
			1. Besprechung Projektauftrag \newline 2. Einarbeitung Thematik \newline 3. Aufsetzen LaTeX-DOkument &
			\begin{minipage}[t]{\linewidth}
				\begin{itemize}[leftmargin=*]
					\item[\Square] MS1: Projektauftrag erhalten
					\item[\Square] Gemeinsame Vision des Projekts 
				\end{itemize}
			\end{minipage} &
			1-2 
		\\ \hline
			E1	& 
			1. Definition der Use Cases \newline 2. Aufbau/Setup/Anordnung \newline 3. Hardware Evaluation \newline 4. Abklären technische Machbarkeit \newline 5. Android Mock-Up \newline 6. Meilensteine und Iterationsplan & 
			\begin{minipage}[t]{\linewidth}
				\begin{itemize}[leftmargin=*]
					\item[\Square] MS2: Review Projektplan
					\item[\Square] Hardware bestellt
					\item[\Square] Mockups für App gezeichnet 
				\end{itemize}
			\end{minipage} &
			3-4
		\\ \hline
			E2	&
			1. Installation openHAB auf \mbox{Raspberry Pi}\newline 2. Einrichten WLAN und Router \newline 3. Use Cases mit DSL umsetzen \newline 4. Integration HomeMatic \newline 5. Integration Philips Hue \newline 6. Integration Webcam \newline & 
			\begin{minipage}[t]{\linewidth}
				\begin{itemize}[leftmargin=*]
					\item[\Square] MS3: Erster Prototyp existiert
				\end{itemize}
			\end{minipage} &
			5-6
		\\ \hline
			E3 	&
			1. Aufsetzen und Anpassen der \mbox{Azure} Cloud \newline 2. Programmierung Azure Binding \newline 3. Integration \mbox{Azure} Cloud \newline 4. Integration Tinkerforge &
		\begin{minipage}[t]{\linewidth}
				\begin{itemize}[leftmargin=*]
					\item[\Square] MS4: Prototyp mit Bindings fertig
					\item[\Square] Cloud aufgesetzt
				\end{itemize}
			\end{minipage} &
			7-8
		\\ \hline
			C1	&
			1. Alle Komponenten vollständig integrieren \newline 2. Vernetzung der Hardware \newline 3. Dokumentation Bindings und Aufbau &
			\begin{minipage}[t]{\linewidth}
				\begin{itemize}[leftmargin=*]
					\item[\Square] MS5: Binding für Cloud erstellt
				\end{itemize}
			\end{minipage} &
			9-10
		\\ \hline
			C2	&
			1. Android App Model portieren (HABDroid) \newline 2. Anbindung Android an Cloud \newline 3. User Interface &
			\begin{minipage}[t]{\linewidth}
				\begin{itemize}[leftmargin=*]
					\item[\Square] MS6: Android-App entwickelt
				\end{itemize}
			\end{minipage} &
			11-12
		\\ \hline
			C3 	&
			1. Refactoring und Unit-Testing \newline 2. Systemtests \newline 3. Überprüfung NFR und FR &
			\begin{minipage}[t]{\linewidth}
				\begin{itemize}[leftmargin=*]
					\item[\Square] MS7: Refactoring und Testing durchgeführt
				\end{itemize}
			\end{minipage} &
			13-14
		\\ \hline
			T1	&
			\tbd &
			\begin{minipage}[t]{\linewidth}
				\begin{itemize}[leftmargin=*]
					\item[\Square] MS8: Abschluss des Projektes
					\item[\Square] Dokumentation abgeschlossen
					\item[\Square] Deliverables übergeben
				\end{itemize}
			\end{minipage} &
			15-16
		\\ \hline
	\end{tabularx}

\subsection*{Besprechungen}
	Wöchentliche Besprechungen: \\

	\begin{tabularx}{\textwidth}{p{3cm}Xlll}
		\textbf{Bezeichnung}	& \textbf{Ziel} & \textbf{Wochentag} & \textbf{Uhrzeit} & \textbf{Ort}
		\\ \hline
			Teambespre-chung &
			Projektarbeiten im Plenum erledigen &
			Donnerstag &
			08:10-08:40	&
			HSR (Labor)
		\\ \hline
			Fortschrittsbe-sprechung &
			Fortschritte bzw. Probleme besprechen &
			Mittwoch &
			10:10-10:50	&
			HSR (6.010)
		\\ \hline
	\end{tabularx}
\pagebreak

\section*{Risikomanagement}
	\subsection*{Risiken}
		Nachstehend wird auf die projektbezogenen Risiken eingegangen. Eine Übersicht in Form einer Tabelle ist 
		auf der nächsten Seite zu finden. Die Tabelle wird während des ganzen Projektes angepasst und aktualisiert, falls notwendig.

	\subsection*{Umgang mit Risiken}
		\subsubsection*{Reserven/Rückstellungen}
			Das grösste Risiko stellt R1 (ungeplante Machbarkeiten) dar. Aus diesem Grund werden in diesem Projekt Rückstellungen von 20 Stunden eingeplant.

		\subsubsection*{Überprüfung von Risiken}
			Weitere Risiken werden im Laufe des Entwicklungsprozesses erkennbar. Hierfür aktualisieren wir dieses Dokument, welches als zentrale Stelle dient, 
			um Entscheidungen und Risiken zu Dokumentieren und auch eine zentrale Anlaufstelle bei Fragen darstellt. Des weiteren wird in der Beschreibung des 
			betroffenen Vorgangs auf mögliche Risiken hingewiesen und dokumentiert.
\pagebreak
	
\begin{landscape}
	\begin{tabularx}{\linewidth}{lp{2cm}Xp{1cm}p{1.5cm}p{1.5cm}XX}
		\textbf{Nr}	& \textbf{Titel} & \textbf{Beschreibung} & \textbf{Scha-den[h]} & \textbf{Eintritts-wahrsch.} & \textbf{Gew. Schad.} & \textbf{Vorbeugung.} & \textbf{Verhalten beim Eintreten.}
		\\ \hline
			\textbf{R1} &
			Ungeplante Machbarkeit &
			Nicht alle Arbeitspakete in Iteration oder Meilensteine abgedeckt. &
			20 & 40\% & 8 & Laufende Kontrolle des Zeitplans &
			Überstunden in Kauf nehmen, um folgende Iteration nicht in Gefahr zu bringen.
		\\ \hline
			\textbf{R2} &
			Absturz Jira-Server und Datenverlust &
			Der virtuelle Server der HSR stürzt ab, und die Daten des Jira gehen verloren. &
			2 &
			10\% &
			0.2 &
			Backup pro Woche erstellen. &
			Letztes Backup einspielen und die Differenz von Hand erneut eintragen.
		\\ \hline
			\textbf{R3} &
			Verlust von Code &
			Das persönliche Notebook stürzt ab und die Daten sind verloren. &
			2 &
			10\% &
			0.2 &
			Code wird ständig auf GitHub gepusht. &
			Lab-PC oder sonstige Computer verwenden und GIT Repository Klonen.
		\\ \hline
			\textbf{R4} &
			Fabrikations-fehler Sensoren &
			Die Sensoren kommen mit einem Fabrikationsfehler an. &
			20 &
			10\% &
			2 &
			  &
			Sensor zurücksenden und mit anderem weiterarbeiten.
		\\ \hline
			\textbf{R5} &
			Schnittstellen Sensoren &
			Schnittstellen zu anderen Systemen bereitet Probleme &
			16 &
			5\% &
			0.8 &
			Dokumentation gut prüfen. &
			Community durchforsten, Workaround suchen.
		\\ \hline
	\end{tabularx}
\end{landscape}

\section*{Arbeitspakete}
	Die Arbeitspakete wurden im Projektmanagementtool Jira als Vorgänge definiert. \\
	Einige Vorgänge beinhalten weitere Untertätigkeiten, die wir ebenfalls als einzelne Arbeitspakete betrachten. \\
	\\
	Eine Übersicht mit allen Arbeitspaketen und dem zeitlichen Ablauf nach Iterationen befindet sich unter: \url{http://sinv-56046.edu.hsr.ch:8080} > Agile > Zeige alle Boards > baIOTBoard > Plan

\section*{Infrastruktur}
	\subsection*{Software}
		Wie in jedem Projekt kommt verschiedene Software zum Einsatz. \\

	\begin{tabularx}{\textwidth}{llX}
		\textbf{Software} & \textbf{Version (Major)} & \textbf{Beschreibung/Einsatzbereich}
		\\ \hline
			GitHub	&
			v3 &
			Source Code Verwaltung inkl. Branchmanagement, Web Interface für Git-Verwaltung.
		\\ \hline
			Atlassian: Jira &
			6.4 &
			Projektmanagement
		\\ \hline
			Windows Server &
			2012 R2 (64Bit) &
			Virtueller Server für Jira
		\\ \hline
			\tbd &
			\tbd &
			\tbd
		\\ \hline
	\end{tabularx}
\pagebreak

\section*{Qualitätsmassnahmen}
	\begin{tabularx}{\textwidth}{XXX}
		\textbf{Massnahme} & \textbf{Zeitraum} & \textbf{Ziel}
		\\ \hline
			Einsetzen eines Projekt-Management-Tools &
			ganzes Projekt &
			Alle auf dem aktuellsten Stand halten
		\\ \hline
			Versionierungssystem (git) ganzes Projekt &
			Sicherung des Codes/Doku, keine Blockaden
		\\ \hline
			Koordinationsmeetings &
			ganzes Projekt &
			Ressourcen optimal zuteilen: Wer benötigt wo Hilfe, wer ist schon fertig?
		\\ \hline
			Vier-Augen-Prinzip &
			ganzes Projekt &
			Dokumentation/Programm-code wird jeweils von beiden Partnern kontrolliert. Bei einem Ausfall einer Person, ist das andere Mitglied informiert.
		\\ \hline
	\end{tabularx}

	\subsection*{Dokumentation}
		\subsubsection*{Ablage}
			Alle Dokumente können auf dem GitHub Repository gefunden werden. Die Vorgänge werden mit Jira auf einem virtuellen Server der HSR verwaltet.
			\begin{itemize}
				\item Dokumentation: \url{https://github.com/greekins/baIOT_TeX} 
				\item Vorgänge: \url{http://sinv-56046.edu.hsr.ch:8080}
			\end{itemize}
			Der Source-Code wird mit Git verwaltet: \tbd

		\subsubsection*{Qualität}
			\begin{itemize}
				\item Commits verlangen eine Beschreibung
				\item Benutzerfreundliche Commit-Übersicht dank Github
				\item Für die Qualität des Codes wird in jeder Iteration (ab Elaboration E2) Codereviews durchgeführt (siehe Managementabläufe)
			\end{itemize}

	\subsection*{Projektmanagement}
		Es wird die von Atlassian zur Verfügung gestellte Umgebung eingesetzt: \\
		\url{http://sinv-56046.edu.hsr.ch:8080} \\
		\textbf{Gast Login:} hhuser


	\subsection*{Entwicklung}
		\subsubsection*{Code Reviews}
			Die Commits sind für alle Projektmitglieder ersichtlich und werden in einem Activity Stream auf dem Repository unter 
			«Graph» angezeigt. Diese werden sporadisch von den anderen Mitgliedern geöffnet und kurz überprüft. \\
			\\
			Bei einem wöchentlichen Meeting werden getätigte Implementierungen im Plenum angeschaut und besprochen. Auch lautet usere 
			Regel, dass bei Unsicherheiten bei laufender Entwicklung Rat vom anderen Teammitglied eingeholt wird.

		\subsubsection*{Code Style Guidelines}
			Es wird sich an die gängigen Style Guidelines gehalten, die im Laufe des Studiums eingeführt wurden.