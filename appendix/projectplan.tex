\chapter{Projektplan}

\begin{center}
	\includegraphics[scale=0.75]{appendix/img/openhabLogo}
\end{center}
\vfill
\begin{description}[style=multiline,leftmargin=3cm]
\item[Thema] Aufbau einer Smart-Home Beispielapplikation
\item[Studenten] Dominik Freier, Marco Leutenegger
\item[Betreuer] Prof. Hansjörg Huser
\end{description}
\pagebreak

\section*{Änderungsgeschichte}
	\begin{tabularx}{\textwidth}{lllX}
	\textbf{Datum}		& \textbf{Version}	& \textbf{Änderung}	& \textbf{Autor} \\
	\hline
	25.02.2015			& 0.0.1				& Dokument erstellen & Marco Leutenegger \\
	\hline
	\tbd					& 0.0.2				& Dokument aktualisieren	& Dominik Freier, Marco Leutenegger \\
	\hline
	\end{tabularx}
\pagebreak

\section*{Einführung}
\subsection*{Zweck}
Dieses Dokument dient als Projektplan für die Bachelorarbeit von Dominik Freier und Marco Leutenegger und definiert alle organisatorischen Rahmenbedingungen.

\subsection*{Gültigkeitsbereich}
Die Gültigkeit des Projektplans beschränkt sich auf die Bachelorarbeit von Dominik Freier und Marco Leutenegger im Frühjahrssemester 2015.

\subsection*{Referenzen}
\begin{tabularx}{\textwidth}{lX}
	\textbf{Bezeichnung}	& \textbf{Referenz} \\
	\hline
	Risikomanagement		& Siehe separates Dokument\\
	\hline
	Security Infos		& \url{https://github.com/openhab/openhab/wiki/Security} \\
	\hline
\end{tabularx}
\pagebreak

\section*{Projekt und Übersicht}
\subsection*{Zweck und Ziel}
Diese Bachelorarbeit hat as Ziel, eine Smart-Home Beispielapplikation aufzubauen, welche wesentliche Aspekte einer Internet-of-Things-Anwendung demonstriert, wie Steuern von Devices, Lesen von Sensoren, Event-Verarbeitung, Überwachung und intelligente Abläufe steuern, Streaming von Sensordaten und Online-Analyse der Daten usw. \\
Das System soll auf einer tragfähigen und erweiterbaren Architektur aufgebaut werden und Microsoft Azure als Cloud Plattform benutzen.

\subsection*{Lieferumfang}
Die abzuliefernden Dokumente und Software-Artefakte des Projekts richten sich im Wesentlichen nach den Vorgaben aus den Dokumentationsanleitungen der HSR. Eine davon abweichender Lieferumfang wurde mit dem Betreuer besprochen und genehmigt.

\subsection*{Referenzen}
\begin{tabularx}{\textwidth}{llXll}
	\textbf{Nr.}	& \textbf{Art} & \textbf{Bezeichnung} & \textbf{Form} & \textbf{Empfänger}\\
	\hline
	1 & Publikation & Poster 							& PDF 			& H.Huser \\	\hline
	2 & Publikation & Kurzfassung 						& PDF 			& H.Huser \\\hline
	3 & Dokument		& Bericht 							& PDF/Ausdruck	& H.Huser \\\hline
	4 & Dokument		& Projektplan 						& PDF/Ausdruck	& H.Huser \\\hline
	5 & Dokument		& Sitzungsprotokolle 				& PDF/Ausdruck	& H.Huser \\\hline
	6 & Dokument		& Eigenständigkeitserklärung			& PDF/Ausdruck	& H.Huser \\\hline
	7 & Dokument		& Erfahrungsbericht D.Freier 		& PDF/Ausdruck	& H.Huser \\\hline
	8 & Dokument		& Erfahrungsbericht M.Leutenegger 	& PDF/Ausdruck	& H.Huser \\\hline
	9 & Source		& Code-Abgabe 						& ZIP			& H.Huser \\\hline
	10& Archiv 		& 2x Deliverables 1-9 				& DVD			& H.Huser \\\hline
\end{tabularx}
\pagebreak


\section*{Projektorganisation}
Die Dokumentation des Projekts gliedert sich in diesen Projektplan und einen Bericht. Im Projektplan werden alle organisatorischen Aspekte festgehalten, wie etwa die Planung der Meilensteine, Aufgaben der Teammitglieder oder Abmachungen zum Dokumentemanagement. Im Bericht werden technische Beschreibungen der Ausgangslage, Diskussionen für Lösungsansätze, Requirements und Details zur Umsetzung dokumentiert. \\
\\
Damit die Teammitglieder möglichst parallel und effizient arbeiten können, werden alle Dokumente mit LaTeX geschrieben und auf einem Git-Repository verwaltet. Daruch wird das Risiko von Versionskonflikten reduziert und der Zugriff insbesondere für den Betreuer vereinfacht. \\
\\
Die Verwaltung der Aufgaben und agilen Vorgänge erfolgt durch Jira. Wir erhielten zu diesem Zweck eine Classroom Lizenz vom Hersteller Atlassian. Jira wurde auf einem virtuellen Server der HSR installiert.

\subsection*{Organisationsstruktur}
\begin{tabularx}{\textwidth}{Xl}
	\textbf{Verantwortung}	& \textbf{Teammitglied} \\
	\hline
	Verwaltung und Bereinigung de Dokumente		& D. Freier, M. Leutenegger \\ \hline
	Pflege virtueller Server, Jira inkl. Backups	& D. Freier, M. Leutenegger \\ \hline
	Sitzungsprotokolle verfassen					& D. Freier, M. Leutenegger \\ \hline
	Iterationsplanung							& D. Freier, M. Leutenegger \\ \hline
	Risikomanagement								& D. Freier, M. Leutenegger \\ \hline
	Architekturdesign							& D. Freier, M. Leutenegger \\ \hline
\end{tabularx}

\subsection*{Externe Schnittstellen}
Betreuer der Bachelorarbeit ist Prof. Hansjörg Huser. Experte ist Herr Stefan Zettel. Gegenleser ist \tbd .
\pagebreak

\section*{Management Abläufe}
\subsection*{Zeitliche Planung}
Das Projekt wird während des Frühjahrssemester 2015 durchgeführt. Der Start der Arbeit war am Montag, den 16. Februar 2015. Die Abgabe der Vollständigen Dokumentation an den Betreuer erfolgt am Freitag, den 12. Juni 2015. Als Zeitbudget sollen in den 17 Wochen insgesamt 720 Stunden, bzw. rund 21 Stunden pro Woche und Student eingeplant werden.

\subsection*{Vorgehensmodell}
Als Vorgehensmodell wurde der Rational Unified Process ausgewählt, da das Projektteam mit diesem Modell aus früheren Arbeiten (inkl. Semesterarbeit) vertraut ist und damit gute Erfahrungen gemacht hat. Die Phasen wurden nach dem Schema "eins, drei, zwei, ein" in insgesamt sieben Iterationen à zwei Wochen aufgeteilt. \tbd

\subsection*{Meilensteine}
\tbd

\subsection*{Iterationsplanung}
\tbd

\subsection*{Besprechungen}
Wöchentliche Besprechungen: \\

\begin{tabularx}{\textwidth}{lXlll}
	\textbf{Bezeichnung}	& \textbf{Ziel} & \textbf{Wochentag} & \textbf{Uhrzeit} & \textbf{Ort}\\
	\hline
	Teambesprechung			& Projektarbeiten im Plenum erledigen	& Donnerstag & 08:10-08:40	& HSR (Labor) \\ \hline
	Fortschrittsbesprechung	& Fortschritte bzw. Probleme besprechen		& Mittwoch & 10:10-10:50	& HSR (6.010) \\ \hline
\end{tabularx}
\pagebreak

\section*{Risikomanagement}
\subsection*{Risiken}
Unser projektbezogenes Risikomanagement wird in einem separaten Excel-Sheet aufgeführt. Nachstehend ist ein Snapshot dieses Sheets zu sehen. Das Dokument wird laufend angepasst und aktualisiert, falls notwendig. \tbd

\subsection*{Umgang mit Risiken}
\subsubsection*{Reserven/Rückstellungen}
Das grösste Risiko stellt R3 (ungeplante Machbarkeiten) \tbd dar. Aus diesem Grund werden in diesem Projekt Rückstellungen von 20 Stunden eingeplant.

\subsubsection*{Überprüfung von Risiken}
Weitere Risiken werden im Laufe des Entwicklungsprozesses erkennbar. Hierfür verwenden wir ein Entscheidungs-Dokument, welches als zentrale Stelle dient, um Entscheidungen und Risiken zu DOkumentieren und eine zentrale Anlaufstelle bei Fragen darstellt. Des weiteren wird in der Beschreibung des betroffenen Vorgangs auf mögliche Risiken hingewiesen und dokumentiert.
\pagebreak

\begin{landscape}
	\begin{tabularx}{\linewidth}{lp{2cm}Xp{1cm}p{1.5cm}p{1.5cm}XX}
	\textbf{Nr}	& \textbf{Titel} & \textbf{Beschreibung} & \textbf{Scha-den[h]} & \textbf{Eintritts-wahrsch.} & \textbf{Gew. Schad.} & \textbf{Vorbeugung.} & \textbf{Verhalten beim Eintreten.}\\
	\hline
	\textbf{R1} & Ungeplante Machbarkeit	& Nicht alle Arbeitspakete in Iteration oder Meilensteine abgedeckt.	& 20 & 40\% & 8 & Laufende Kontrolle des Zeitplans & Überstunden in Kauf nehmen, um folgende Iteration nicht in Gefahr zu bringen. \\ \hline
	\textbf{R2} & Absturz Jira-Server und Datenverlust	& Der virtuelle Server der HSR stürzt ab, und die Daten des Jira gehen verloren. & 2 & 10\% & 0.2 & Backup pro Woche erstellen. & Letztes Backup einspielen und die Differenz von Hand erneut eintragen. \\ \hline
	\textbf{R3} & Verlust von Code & Das persönliche Notebook stürzt ab und die Daten sind verloren. & 2 & 10\% & 0.2 & Code wird ständig auf GitHub gepusht. & Lab-PC oder sonstige Computer verwenden und GIT Repository Klonen. \\ \hline
	\textbf{R4} & Fabrikations-fehler Sensoren & Die Sensoren kommen mit einem Fabrikationsfehler an. & 20 & 10\% & 2 & & Sensor zurücksenden und mit anderem weiterarbeiten. \\ \hline
	\textbf{R5} & Schnittstellen Sensoren & Schnittstellen zu anderen Systemen bereitet Probleme & 16 & 30\% & 4.8 & Dokumentation gut prüfen. & Community durchforsten, Workaround suchen. \\ \hline
\end{tabularx}
\end{landscape}

\section*{Arbeitspakete}
Die Arbeitspakete wurden im Projektmanagementtool Jira als Vorgänge definiert. \\
Einige Vorgänge beinhalten weitere Untertätigkeiten, die wir ebenfalls als einzelne Arbeitspakete betrachten. \\
\\
Eine Übersicht mit allen Arbeitspaketen und dem zeitlichen Ablauf nach Iterationen befindet sich unter: \url{http://sinv-56046.edu.hsr.ch:8080} > Agile > Zeige alle Boards > baIOTBoard > Plan

\section*{Infrastruktur}
\subsection*{Software}
Wie in jedem Projekt kommt verschiedene Software zum Einsatz. \\

\begin{tabularx}{\textwidth}{llX}
	\textbf{Software}	& \textbf{Version (Major)} & \textbf{Beschreibung/Einsatzbereich} \\ \hline
	GitHub	& v3 & Source Code Verwaltung inkl. Branchmanagement, Web Interface für Git-Verwaltung. \\ \hline
	Atlassian: Jira & 6.4 & Projektmanagement \\ \hline
	Windows Server	& 2012 R2 (64Bit) & Virtueller Server für Jira \\ \hline
	\tbd & \tbd & \tbd \\ \hline
\end{tabularx}
\pagebreak

\section*{Qualitätsmassnahmen}

\begin{tabularx}{\textwidth}{XXX}
	\textbf{Massnahme}	& \textbf{Zeitraum} & \textbf{Ziel} \\ \hline
	Einsetzen eines Projekt-Management-Tools & ganzes Projekt & Alle auf dem aktuellsten Stand halten \\ \hline
	Versionierungssystem (git) ganzes Projekt & Sicherung des Codes/Doku, keine Blockaden \\ \hline
	Koordinationsmeetings & ganzes Projekt & Ressourcen optimal zuteilen: Wer benötigt wo Hilfe, wer ist schon fertig? \\ \hline
	Vier-Augen-Prinzip & ganzes Projekt & Dokumentation/Programm-code wird jeweils von beiden Partnern kontrolliert. Bei einem Ausfall einer Person, ist das andere Mitglied informiert. \\ \hline
\end{tabularx}

\subsection*{Dokumentation}
\subsubsection*{Ablage}
Alle Dokumente können auf dem GitHub Repository gefunden werden. Die Vorgänge werden mit Jira auf einem virtuellen Server der HSR verwaltet.
\begin{itemize}
	\item Dokumentation: \url{https://github.com/greekins/baIOT_TeX} 
	\item Vorgänge: \url{http://sinv-56046.edu.hsr.ch:8080}
\end{itemize}
Der Source-Code wird mit Git verwaltet: \tbd

\subsubsection*{Qualität}
\begin{itemize}
	\item Commits verlangen eine Beschreibung
	\item Benutzerfreundliche Commit-Übersicht dank Github
	\item Für die Qualität des Codes wird in jeder Iteration (ab Elaboration E2) Codereviews durchgeführt (siehe Managementabläufe)
\end{itemize}

\subsection*{Projektmanagement}
Es wird die von Atlassian zur Verfügung gestellte Umgebung eingesetzt: \\
\url{http://sinv-56046.edu.hsr.ch:8080} \\
\textbf{Gast Login:} hhuser


\subsection*{Entwicklung}
\subsubsection*{Code Reviews}
Die Commits sind für alle Projektmitglieder ersichtlich und werden in einem Activity Stream auf dem Repository unter «Graph» angezeigt. Diese werden sporadisch von den anderen Mitgliedern geöffnet und kurz überprüft. \\
\\
Bei einem wöchentlichen Meeting werden getätigte Implementierungen im Plenum angeschaut und besprochen. Auch lautet usere Regel, dass bei Unsicherheiten bei laufender Entwicklung Rat vom anderen Teammitglied eingeholt wird.

\subsubsection*{Code Style Guidelines}
Es wird sich an die gängigen Style Guidelines gehalten, die im Laufe des Studiums eingeführt wurden.