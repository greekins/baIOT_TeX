\chapter{Projektplan}

\includegraphics{appendix/img/openhabLogo}

\begin{description}[style=multiline,leftmargin=3cm]
\item[Thema] Aufbau einer Smart-Home Beispielapplikation
\item[Studenten] Dominik Freier, Marco Leutenegger
\item[Betreuer] Prof. Hansjörg Huser
\end{description}
\pagebreak

\section*{Änderungsgeschichte}
	\begin{tabularx}{\textwidth}{lllX}
	\textbf{Datum}		& \textbf{Version}	& \textbf{Änderung}	& \textbf{Autor} \\
	\hline
	25.02.2015			& 0.0.1				& Dokument erstellen & Marco Leutenegger \\
	\hline
	\tbd					& 0.0.2				& Dokument aktualisieren	& Dominik Freier, Marco Leutenegger \\
	\hline
	\end{tabularx}
\pagebreak

\section*{Einführung}
\subsection*{Zweck}
Dieses Dokument dient als Projektplan für die Bachelorarbeit von Dominik Freier und Marco Leutenegger und definiert alle organisatorischen Rahmenbedingungen.

\subsection*{Gültigkeitsbereich}
Die Gültigkeit des Projektplans beschränkt sich auf die Bachelorarbeit von Dominik Freier und Marco Leutenegger im Frühjahrssemester 2015.

\subsection*{Referenzen}
\begin{tabularx}{\textwidth}{lX}
	\textbf{Bezeichnung}	& \textbf{Referenz} \\
	\hline
	Risikomanagement		& Siehe separates Dokument\\
	\hline
	Security Infos		& \url{https://github.com/openhab/openhab/wiki/Security} \\
	\hline
\end{tabularx}
\pagebreak

\section*{Projekt und Übersicht}
\subsection*{Zweck und Ziel}
Diese Bachelorarbeit hat as Ziel, eine Smart-Home Beispielapplikation aufzubauen, welche wesentliche Aspekte einer Internet-of-Things-Anwendung demonstriert, wie Steuern von Devices, Lesen von Sensoren, Event-Verarbeitung, Überwachung und intelligente Abläufe steuern, Streaming von Sensordaten und Online-Analyse der Daten usw. \\
Das System soll auf einer tragfähigen und erweiterbaren Architektur aufgebaut werden und Microsoft Azure als Cloud Plattform benutzen.

\subsection*{Lieferumfang}
Die abzuliefernden Dokumente und Software-Artefakte des Projekts richten sich im Wesentlichen nach den Vorgaben aus den Dokumentationsanleitungen der HSR. Eine davon abweichender Lieferumfang wurde mit dem Betreuer besprochen und genehmigt.

\subsection*{Referenzen}
\begin{tabularx}{\textwidth}{llXll}
	\textbf{Nr.}	& \textbf{Art} & \textbf{Bezeichnung} & \textbf{Form} & \textbf{Empfänger}\\
	\hline
	1 & Publikation & Poster 							& PDF 			& H.Huser \\	\hline
	2 & Publikation & Kurzfassung 						& PDF 			& H.Huser \\\hline
	3 & Dokument		& Bericht 							& PDF/Ausdruck	& H.Huser \\\hline
	4 & Dokument		& Projektplan 						& PDF/Ausdruck	& H.Huser \\\hline
	5 & Dokument		& Sitzungsprotokolle 				& PDF/Ausdruck	& H.Huser \\\hline
	6 & Dokument		& Eigenständigkeitserklärung			& PDF/Ausdruck	& H.Huser \\\hline
	7 & Dokument		& Erfahrungsbericht D.Freier 		& PDF/Ausdruck	& H.Huser \\\hline
	8 & Dokument		& Erfahrungsbericht M.Leutenegger 	& PDF/Ausdruck	& H.Huser \\\hline
	9 & Source		& Code-Abgabe 						& ZIP			& H.Huser \\\hline
	10& Archiv 		& 2x Deliverables 1-9 				& DVD			& H.Huser \\\hline
\end{tabularx}
\pagebreak


\section*{Projektorganisation}
\tbd

\subsection*{Organisationsstruktur}
\tbd

\subsection*{Externe Schnittstellen}
\tbd

\section*{Management Abläufe}
\subsection*{Zeitliche Planung}
\tbd

\subsection*{Vorgehensmodell}
\tbd

\subsection*{Meilensteine}
\tbd

\subsection*{Iterationsplanung}
\tbd

\subsection*{Besprechungen}
\tbd

\section*{Risikomanagement}
\subsection*{Risiken}
\tbd

\subsection*{Umgang mit Risiken}
\subsubsection*{Reserven / Rückstellungen}
\tbd

\subsubsection*{Überprüfung von Risiken}
\tbd

\section*{Arbeitspakete}
\tbd

\section*{Infrastruktur}
\subsection*{Software}
\tbd

\section*{Qualitätsmassnahmen}
\tbd

\subsection*{Dokumentation}
\subsubsection*{Ablage}
\tbd

\subsubsection*{Qualität}
\tbd

\subsection*{Projektmanagement}
\tbd

\subsection*{Entwicklung}
\subsubsection*{Code Reviews}
\tbd

\subsubsection*{Code Style Guidelines}